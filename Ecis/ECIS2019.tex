\documentclass[]{elsarticle} %review=doublespace preprint=single 5p=2 column
%%% Begin My package additions %%%%%%%%%%%%%%%%%%%
\usepackage[hyphens]{url}

  \journal{ECIS 2019} % Sets Journal name


\usepackage{lineno} % add
\providecommand{\tightlist}{%
  \setlength{\itemsep}{0pt}\setlength{\parskip}{0pt}}

\bibliographystyle{elsarticle-harv}
\biboptions{sort&compress} % For natbib
\usepackage{graphicx}
\usepackage{booktabs} % book-quality tables
%%%%%%%%%%%%%%%% end my additions to header

\usepackage[T1]{fontenc}
\usepackage{lmodern}
\usepackage{amssymb,amsmath}
\usepackage{ifxetex,ifluatex}
\usepackage{fixltx2e} % provides \textsubscript
% use upquote if available, for straight quotes in verbatim environments
\IfFileExists{upquote.sty}{\usepackage{upquote}}{}
\ifnum 0\ifxetex 1\fi\ifluatex 1\fi=0 % if pdftex
  \usepackage[utf8]{inputenc}
\else % if luatex or xelatex
  \usepackage{fontspec}
  \ifxetex
    \usepackage{xltxtra,xunicode}
  \fi
  \defaultfontfeatures{Mapping=tex-text,Scale=MatchLowercase}
  \newcommand{\euro}{€}
\fi
% use microtype if available
\IfFileExists{microtype.sty}{\usepackage{microtype}}{}
\usepackage{color}
\usepackage{fancyvrb}
\newcommand{\VerbBar}{|}
\newcommand{\VERB}{\Verb[commandchars=\\\{\}]}
\DefineVerbatimEnvironment{Highlighting}{Verbatim}{commandchars=\\\{\}}
% Add ',fontsize=\small' for more characters per line
\usepackage{framed}
\definecolor{shadecolor}{RGB}{248,248,248}
\newenvironment{Shaded}{\begin{snugshade}}{\end{snugshade}}
\newcommand{\KeywordTok}[1]{\textcolor[rgb]{0.13,0.29,0.53}{\textbf{#1}}}
\newcommand{\DataTypeTok}[1]{\textcolor[rgb]{0.13,0.29,0.53}{#1}}
\newcommand{\DecValTok}[1]{\textcolor[rgb]{0.00,0.00,0.81}{#1}}
\newcommand{\BaseNTok}[1]{\textcolor[rgb]{0.00,0.00,0.81}{#1}}
\newcommand{\FloatTok}[1]{\textcolor[rgb]{0.00,0.00,0.81}{#1}}
\newcommand{\ConstantTok}[1]{\textcolor[rgb]{0.00,0.00,0.00}{#1}}
\newcommand{\CharTok}[1]{\textcolor[rgb]{0.31,0.60,0.02}{#1}}
\newcommand{\SpecialCharTok}[1]{\textcolor[rgb]{0.00,0.00,0.00}{#1}}
\newcommand{\StringTok}[1]{\textcolor[rgb]{0.31,0.60,0.02}{#1}}
\newcommand{\VerbatimStringTok}[1]{\textcolor[rgb]{0.31,0.60,0.02}{#1}}
\newcommand{\SpecialStringTok}[1]{\textcolor[rgb]{0.31,0.60,0.02}{#1}}
\newcommand{\ImportTok}[1]{#1}
\newcommand{\CommentTok}[1]{\textcolor[rgb]{0.56,0.35,0.01}{\textit{#1}}}
\newcommand{\DocumentationTok}[1]{\textcolor[rgb]{0.56,0.35,0.01}{\textbf{\textit{#1}}}}
\newcommand{\AnnotationTok}[1]{\textcolor[rgb]{0.56,0.35,0.01}{\textbf{\textit{#1}}}}
\newcommand{\CommentVarTok}[1]{\textcolor[rgb]{0.56,0.35,0.01}{\textbf{\textit{#1}}}}
\newcommand{\OtherTok}[1]{\textcolor[rgb]{0.56,0.35,0.01}{#1}}
\newcommand{\FunctionTok}[1]{\textcolor[rgb]{0.00,0.00,0.00}{#1}}
\newcommand{\VariableTok}[1]{\textcolor[rgb]{0.00,0.00,0.00}{#1}}
\newcommand{\ControlFlowTok}[1]{\textcolor[rgb]{0.13,0.29,0.53}{\textbf{#1}}}
\newcommand{\OperatorTok}[1]{\textcolor[rgb]{0.81,0.36,0.00}{\textbf{#1}}}
\newcommand{\BuiltInTok}[1]{#1}
\newcommand{\ExtensionTok}[1]{#1}
\newcommand{\PreprocessorTok}[1]{\textcolor[rgb]{0.56,0.35,0.01}{\textit{#1}}}
\newcommand{\AttributeTok}[1]{\textcolor[rgb]{0.77,0.63,0.00}{#1}}
\newcommand{\RegionMarkerTok}[1]{#1}
\newcommand{\InformationTok}[1]{\textcolor[rgb]{0.56,0.35,0.01}{\textbf{\textit{#1}}}}
\newcommand{\WarningTok}[1]{\textcolor[rgb]{0.56,0.35,0.01}{\textbf{\textit{#1}}}}
\newcommand{\AlertTok}[1]{\textcolor[rgb]{0.94,0.16,0.16}{#1}}
\newcommand{\ErrorTok}[1]{\textcolor[rgb]{0.64,0.00,0.00}{\textbf{#1}}}
\newcommand{\NormalTok}[1]{#1}
\ifxetex
  \usepackage[setpagesize=false, % page size defined by xetex
              unicode=false, % unicode breaks when used with xetex
              xetex]{hyperref}
\else
  \usepackage[unicode=true]{hyperref}
\fi
\hypersetup{breaklinks=true,
            bookmarks=true,
            pdfauthor={},
            pdftitle={Coordinating Open Innovation through Tokenization: A Methodology},
            colorlinks=true,
            urlcolor=blue,
            linkcolor=magenta,
            pdfborder={0 0 0}}
\urlstyle{same}  % don't use monospace font for urls

\setcounter{secnumdepth}{0}
% Pandoc toggle for numbering sections (defaults to be off)
\setcounter{secnumdepth}{0}
% Pandoc header



\begin{document}
\begin{frontmatter}

  \title{Coordinating Open Innovation through Tokenization: A Methodology}
    \author[University]{Author 01\corref{c1}}
   \ead{email@email.com} 
   \cortext[c1]{Corresponding Author}
    \author[University]{Author 02}
   \ead{email@email.com} 
  
      \address[University 01]{Department, Street, City, State, Zip}
    \address[University 02]{Department, Street, City, State, Zip}
  
  \begin{abstract}
  This is the abstract. It consists of two paragraphs.
  \end{abstract}
  
 \end{frontmatter}

\section{1 Introduction}\label{introduction}

This paper presents the preliminary results of an ongoing study that
aims at designing new technological solutions for data storage, such as
blockchain, to allow firms to perform open innovation by means of
decentralized idea challenges. Open innovation is \emph{a paradigm that
assumes that firms can and should use external ideas as well as internal
ideas, and internal and external paths to market, as the firms look to
advance their technology} (Chesbrough, Vanhaverbeke, and West 2006;
Vanhaverbeke and Chesbrough 2014). Among the different technologies that
have enabled open innovation, blockchain could actually be considered
those that instantiate open innovation in an infrastructure for
transactions, products, and services (Shrier, Wu, and Pentland 2016).
Risen as the technology paradigm on which the bitcoin relied and spread
(Nakamoto 2008), blockchain has been receiving increasing attention as
the basis for applications beyond cryptocurrencies (D. Tapscott 2017).
This is due to the individual relevance of different technologies making
it up, such as smart contracts, encryption, and distributed Ledger
(Halaburda 2018).\\
Taking these issues into account, a blockchain generally defined as a
\emph{distributed ledger of transactions} (Halaburda 2018) is a
continuously growing list of records, called blocks, which are linked
and secured using cryptography. Each block typically contains a
cryptographic hash of the previous block, a timestamp, and transaction
data (Swan 2015; D. Tapscott and Tapscott 2016). Blockchain allows
tokenization, \emph{the process of substituting a sensitive data element
with a non-sensitive equivalent, referred to as a token, that has no
extrinsic or exploitable meaning or value}.\\
In this article we assess how to use blockchain to perform tokenization
of content and to manage idea challenges in a decentralized fashion, in
order to increase the number of innovative ideas, defined here as
\emph{ideas that differ from each other and meet the requirements of the
solution seeker}.\\
Thus, our research question is: \textbf{how to use tokenization to
manage idea challenges in a decentralized fashion?}\\
The rest of the paper proceeds as it follows. In section 2 we briefly
review the existing literature and we identify the relevant concepts
needed to answer our research question. Section 3 summarizes how we used
design science to create an artefact under the shape of a method,
whereas section 4 describes how we tested it by means of simulation.
Section 5 concludes the paper by discussing its limitation and by
suggesting further directions of investigation.

\section{2 Background and motivations}\label{background-and-motivations}

In this section, we briefly review existing literature on management
literature concerning different solutions to decentralize the innovation
process.\\
In the last decade, open innovation (Chesbrough and Borgers 2014) has
been more and more considered by private as well as public organizations
(Viscusi, Poulin, and Tucci 2015) as a way to obtain competitive
advantage through the exploitation of the opportunities and capabilities
offered by digitalization (Tilson, Lyytinen, and Sørensen 2010; Yoo
2013) and exploration of alternative solutions to research and
development. Among the phenomena related to open innovation the
tokenization of work and the adoption of collective contests for ideas
searching and problem solving through crowdsourcing (Afuah and Tucci
2012; Tucci, Afuah, and Viscusi 2018; Boudreau and Lakhani 2013; Felin,
Lakhani, and Tushman 2017) received an increasing attention as a way to
exploit collective intelligence for innovation as well as to improve
performance of organizations (Malone and Bernstein 2015; Woolley,
Aggarwal, and Malone 2015 ,Riedl and Woolley (2017)). Taking these
issues into account, web-based idea competitions such as Innocentive has
already been proven successful at gathering a large set of solvers to
address complex problems (K. Lakhani, Lifshitz-Assaf, and Tushman 2012).
Bullinger et al. (2010) have reviewed 52 publications on ideas
competition and identified 10 design elements for ideas competition,
noticing the most common combination is: an (1) online idea competition
(2) initiated by a firm (3) that sets a medium amount of specificity
concerning the required tasks, which can range from (4) ideas, sketches,
concepts, prototype, and working solution, for a (5) specific target
group of (6) individuals that (7) competes and cooperates thanks to
community functions in the online platform, for (8) a large amount of
time, (9) in exchange of a mix between monetary and non-monetary rewards
that are given by (10) a jury of experts. With respect to the process
used to generate ideas, we refer to Faste et al. (2013), who compared
classic brainstorming with two new solutions: (a) chainstorming, where
participants have to use ideas from a previous brainstorming to solve a
new problem and (b) cheatstorming, where participants cannot use new
ideas and can only use ideas from previous brainstorming. Since
performance of cheatstorming seems to outperform the other options, the
authors suggested that idea generation is less about idea and more about
dealing with cultural influence exerted by unconventional ideas on the
ideating team. Therefore, tokenization could be used to list the ideas
submitted and rewards those that are winning as well as those that are
reused, as long as a way to regroup similar ideas in a decentralized
fashion.

\section{3 Methodology}\label{methodology}

In this section we illustrate the chosen methodology to answer our
research question. We position our study in the field of design science
research (A. Hevner et al. 2004) and we developed an artefact under the
shape of a method, as defined by S. T. March and Smith (1995).
Accordingly, we have followed the guidelines of Gregor and Hevner (2013)
to create a set of design principles as part of a nascent design theory,
which is at level 2 of the contribution types. Finally, we have selected
simulation as an experimental way to test our assumptions ex-ante in a
setting that allowed to control the variables. Artificial evaluation can
be seen as unreal due to (a) unreal users, (b) unreal systems, and (c)
unreal problems since it is not held by the users and/or not real tasks,
etc.(Sun and Kantor 2006). Nonetheless, Pries-Heje, Baskerville, and
Venable (2008; Venable, Pries-Heje, and Baskerville 2012) recommend to
use this type of artificial evaluation as an ex-ante assessment of the
system, before testing it on larger scale with real users (naturalistic
evaluations).\\
In what follows we present our artefact in the shape of a method
composed of four phases, as shown in figure 1, and we illustrate how it
works by a simple example.

Table 1 illustrates the results of an idea competition concerning how to
reduce energy consumption in private households. For sake of simplicity,
we will assume to obtain four ideas. Each idea receives a unique
identification number, which is associated with the unique ID of the
user (Step 1). Then, by means of automatic or manual classification, the
ideas are assigned to clusters showing the ``stance'' of each idea: for
example, ``behavioural'' vs. ``technical'' or ``engineering'' oriented
idea (Step 2). For sake of discussion, we can also assume that a set of
evaluators give a score between 1 and 9 for each idea according to three
dimensions, which have different weights (D1 is 100, D2 is 10 and D3 is
1). Finally, the first idea of each cluster is selected, the selected
players (065, 177 and 356) win a prize and are invited for phase 2.\\
In the second phase, the players perform chainstorm and combine the
ideas that passed phase 01, to produce new ideas. Table 2 shows the
creation of two new ideas. In this step, the ideas are assessed by
evaluators, who give a score between 1 and 9 for each idea according to
three dimensions, which have different weights (D1 is 100, D2 is 10 and
D3 is 1). For sake of illustration, we can assume that the score of two
ideas in one dimension is the sum of the scores of the two individual
ideas in that dimension. That allows us to predict the score of idea 005
and idea 006, and choose idea 686, since it maximizes the satisfaction
of the client.\\
A standard idea challenge would have rewarded the best idea (in this
case idea 003, with a score of 724). Nonetheless, that would require an
expert assessment of all ideas, whereas our model allows assessing only
the pooled ideas, which are the sum of possible combinations among ideas
selected in phase 01. Hence, ideas to assess =
\{{[}1+(clusters/2){]}*(cluster/2){]}/2 That leads to the first design
guideline for a decentralized idea challenge, which is associated with a
testable proposition.\\
\textbf{P1}: automatic clustering of tokenized ideas in a blockchain and
random selection significantly reduce the number of ideas to assess but
it decreases the quality of the outcome.

Moreover, one could expect the client to seek for a solution that scores
9 in every dimension, which leads to 999 as target in our example. Thus,
it is possible to estimate the difference between the target and the
chosen idea in phase 02, which is 999 -- 687 = 312. Setting 312 as the
new target, we could select the ideas from phase 01 that are below the
target, in our case idea 002 and idea 004. Then, we could set up a
cheatstorming competition that rewards all ideas below the target for
contribution, and then select the best pool of ideas created on a
previous context. In our case, the combination of idea 002 and idea 004
reaches 312. This idea would have not be retained in phase 02, but it
now perfectly fits the narrow problem of the client. Moreover, the
number of pooled ideas might be significant in this step, but it still
below the number of ideas to assess by experts in a standard idea
challenge.\\
That leads to the second design guideline for a decentralized idea
challenge, which is associated with a testable proposition.\\
\textbf{P2}: collection of tokenized ideas in a blockchain done in a
previous challenge increases the quality of the outcome but it increases
the number of ideas to assess

\section{4 Preliminary results by
simulation}\label{preliminary-results-by-simulation}

We use R Statistical package Team (2013) to write an agent-based
simulation, that is a computerized simulation of a number of
decision-makers (agents), which interact through prescribed rules
(Farmer and Foley 2009). Our code has four functions, corresponding to
the following phases:

\begin{itemize}
\tightlist
\item
  phase 00 generates the agents and it assigns one or more ideas to each
  agent. It takes as initial parameters the whole set of possible market
  dimensions assessed by the client (=50), the whole set of possible
  market dimensions assessed by the client (=50), the number of clients
  (=50), the number of experts (=50), the number of agents (=50), the
  maximum skill level per dimension (=100) and the number of challenges
  (=100). In the previous example, the target was 10\^{}3, there were 3
  agents that generated 4 ideas into 3 clusters in phase 01. Our
  simulation uses 10\^{}9 as target, 10\^{}3 agents, 10\^{}4 ideas and
  10 clusters.
\end{itemize}

\begin{verbatim}
##   Agent_ID s_pro_01 s_pro_02 s_pro_03 s_mkt_01 s_mktt_02 s_mkt_03 money
## 1        1        8        9        8        9        11        8     0
## 2        2       10       12        8        8        15       11     0
## 3        3        7        9        9        9         9       10     0
## 4        4       11       12       11       12         7        5     0
## 5        5       14       10        9       16         9       11     0
## 6        6       12        7        8       13        12        7     0
\end{verbatim}

\begin{verbatim}
##   Expert_ID s_pro_01 s_pro_02 s_pro_03 s_mkt_01 s_mktt_02 s_mkt_03 money
## 1         1       51       36       49        8         9        6     0
## 2         2       51       47       46       15         7       11     0
## 3         3       43       50       38        7         3       13     0
## 4         4       60       59       48        9        14        7     0
## 5         5       54       54       54       10        11        5     0
## 6         6       52       50       56       12        15        8     0
\end{verbatim}

\begin{verbatim}
##   Client_ID s_pro_01 s_pro_02 s_pro_03 s_mkt_01 s_mktt_02 s_mkt_03 money
## 1         1        7       12        7       51        48       45     0
## 2         2        7        5        3       55        50       40     0
## 3         3        7       12        5       43        63       45     0
## 4         4        6       16       12       56        40       47     0
## 5         5        7        6       14       51        53       51     0
## 6         6       12        9        8       51        56       50     0
\end{verbatim}

\begin{itemize}
\tightlist
\item
  phase 01 divides the ideas into clusters and retains one idea per
  cluster. We shall compare the case when the winner for each cluster is
  the first one (model A) or the one with the idea that has the greatest
  value (model B).
\item
  phase 02 selects and combines the retained idea into the best idea
  pool. This is a problem of linear programming that tries to maximizes
  the sum of the ideas by defining that the coefficient of each idea is
  binary (either we select it or not) and by setting one constraint (the
  sum of the pooled idea cannot be greater than the target).
\item
  phase 03 assesses the average payoff of agents, by using the mean of
  the individual payoff, and the payoff of the client, by calculating
  the remainder between the target and the pooled ideas. For sake of
  clarity, we shall compare the situation when the idea challenge has 1
  stage and one cluster (Model 0), when we add a second stage performing
  chainstorming (Model 1) and when we add a third stage performing
  cheatstorming.
\end{itemize}

Table 3 illustrates the outcome of the simulations. The first model (M0)
allows to obtain a good idea selected among 10\^{}4 ideas. That assures
a good satisfaction for the client and a small cost in terms of prize
for the only winner. Nonetheless, one can assume it would be fairly
expensive to pay experts to evaluate the 10\^{}4 ideas and it would be
quite disappointing for the agents to have 1 chance out of 10\^{}4 to
win. Model 1a has 10 winners for the phase 01, that is one per cluster.
One can assume that cluster definition could be done by an unsupervised
algorithm for clustering based on textual description of the solution,
so there will be no need to request expert evaluations. In the second
stage, the ideas are pooled together and the outcome is below the one of
the first model, and the agents to reward are 14 (10 in the first stage
and 4 in the second stage). Nonetheless, the cost for idea assessment
are significantly lower, since there are only 15 possible combinations
to evaluate. This validates our first testable proposition (P1).
Moreover, Model 1b appears evident why it would not be wise to select
the best idea for every cluster. Indeed, it would take the same amount
of assessment as the same model to obtain the same result (in the second
phase, the agent with the best idea cannot find anyone that can improve
it).

Finally, model 2 has been conceived to illustrate that with a large set
of agents it would be theoretically possible to receive a number of
ideas to assess that is greater than the initial number, although it is
unlikely to occur in reality. This validates our second testable
proposition (P2) under the condition that the number of retained ideas
in the second idea challenge is not too large (although, one could
choose to solve this problem by using idea clustering again).

\section{5 Discussions}\label{discussions}

In this section we assess our article by using the six guidelines for a
design science paper of (A. Hevner et al. 2004). The first guideline
states that design-science research must produce a viable artifact in
the form of a construct, a model, a method, or an instantiation: our
artefact is a model for a decentralized idea challenge, which uses
tokenization in a blockchain to store ideas. The second guideline claims
that the objective of design-science research is to develop
technology-based solutions to important and relevant business problems.
Our solution is based on a new technology and it addresses an increasing
need for firms that seek to innovate. The third guideline reminds that
the utility, quality, and efficacy of a design artifact must be
rigorously demonstrated via well-executed evaluation methods. We have
chosen to use agent-based simulation as a form of ex-ante artificial
assessment, being aware of its advantages and shortcomings. The fourth
guideline says that effective design-science research must provide clear
and verifiable contributions in the areas of the design artifact, design
foundations, and/or design methodologies. Our artefact builds on
previous research on idea challenges and it extends it by offering
design guidelines to implement it in a business context. The fifth
guidelines suggests that design-science research relies upon the
application of rigorous methods in both the construction and evaluation
of the design artifact. In this paper we have illustrated the kernel
theories that led to our solution and the underlying assumption in our
evaluation. The sixth guideline concerns the search for an effective
artifact that requires utilizing available means to reach desired ends
while satisfying laws in the problem environment. Although the
simulation has used a large set of agents to test the boundary
conditions of our design theory, while building our model, we took into
account the tools that a firm would have to create an idea challenge.
Finally, the seventh guideline justify our choice of targeting this
conference, since design-science research must be presented effectively
both to technology-oriented as well as management-oriented audiences.
This research project is currently ongoing and it has two major
limitations. On the one hand, future research shall evaluate our model
in a real context, by performing experimental tests with people solving
the same problem while using different techniques. On the other hand, in
a future paper we shall describe how to perform automatic idea
clustering of the ideas by means of unsupervised algorithm that analyses
the corpora of all the idea and performs text analysis. Nonetheless, our
model already offers to small and big firms the possibility to exploit a
new technology to manage idea challenges in an effective and efficient
way, and it opens new directions for research, such as combining game
theory to assess the payoff of agents playing different theories as well
as exploring the other features of blockchain to improve the performance
of open innovation.

\section{6 Conclusion}\label{conclusion}

In this paper we have proposed a method for exploiting tokenization as a
way to identify, select, and create new ideas from contests, challenges,
and crowdsourcing. The current proposals present a set of limitations,
whose solution will be explored in future work. In particular, the
assignment of values to ideas would require further investigation.
Actually, imagining that each idea receives a grade of 1 to 9 on three
dimensions, and the three dimensions have different weights (100, 10 and
1), one can describe the value of the idea with numbers from 1 to 1000.
The target will always be 9 * 10 \^{} n is the number of dimensions,
because the customer would like to have everything at its best.
Furthermore, in this paper we have considered 10 \^{} n + 1 to model the
fact that the customer will never be satisfied. Another limitation is
that the artefact is still at experimental level and the number of
tested results is still too small for both justifying it and to have an
explicit theory from it. As to this issue, in future work we are going
to explore too different solutions that are, 1) to divide ideas into
clusters (with a clustering algorithm) and take the first idea of each
cluster, which allows to reduce the costs of evaluation drastically, and
2) developing and analyzing a multiple-step challenge, which can
increase the payoff for agents, at limited costs for the customer.
Notwithstanding the current limitations, we believe that the method
would represent a first step towards a full understanding of how to
exploit the opportunities offered by technologies such as blockchain for
open innovation, thus bridging the gap between technical and managerial
perspectives characterizing the two state of the art stances with regard
the two topics.

\section{7 Annexes: R code of the agent-based
simulation}\label{annexes-r-code-of-the-agent-based-simulation}

\begin{Shaded}
\begin{Highlighting}[]
\CommentTok{# Setting up a function to generate ideas}
\NormalTok{phase00 <-}\StringTok{ }\ControlFlowTok{function}\NormalTok{(target, totAgents, totIdeas, clusters) \{}
\KeywordTok{set.seed}\NormalTok{(}\DecValTok{1}\NormalTok{) }\CommentTok{#Setting a seed to allow comparable results }
\CommentTok{# this allows multiple ideas to one owner}
\NormalTok{agentsId <-}
\KeywordTok{as.integer}\NormalTok{(}\KeywordTok{rnorm}\NormalTok{(totIdeas, totAgents }\OperatorTok{/}\StringTok{ }\DecValTok{2}\NormalTok{, totAgents }\OperatorTok{/}\StringTok{ }\DecValTok{10}\NormalTok{)) }
\KeywordTok{set.seed}\NormalTok{(}\DecValTok{1}\NormalTok{) }\CommentTok{#Setting a seed again}
\CommentTok{# Generating random numbers in a uniform distribution}
\NormalTok{idea <-}
\KeywordTok{as.integer}\NormalTok{(}\KeywordTok{runif}\NormalTok{(totIdeas, }\DataTypeTok{min =} \DecValTok{0}\NormalTok{, }\DataTypeTok{max =}\NormalTok{ target)) }
\CommentTok{#Extracting the cluster by dividing each number to ... }
\CommentTok{# ... the number of clusters}
\NormalTok{ideaClusters <-}
\DecValTok{1} \OperatorTok{+}\StringTok{ }\KeywordTok{as.integer}\NormalTok{(idea }\OperatorTok{/}\StringTok{ }\NormalTok{(target }\OperatorTok{/}\StringTok{ }\NormalTok{clusters)) }
\NormalTok{agentsReward_}\DecValTok{01}\NormalTok{ <-}
\KeywordTok{rep}\NormalTok{(}\DecValTok{0}\NormalTok{, totIdeas) }\CommentTok{# A variable associated to phase 01}
\NormalTok{agentsReward_}\DecValTok{02}\NormalTok{ <-}
\KeywordTok{rep}\NormalTok{(}\DecValTok{0}\NormalTok{, totIdeas) }\CommentTok{# A variable associated to phase 02}
\CommentTok{#Creating a dataframe with ID, value and cluster}
\NormalTok{ideas <-}
\KeywordTok{data.frame}\NormalTok{(agentsId,}
\NormalTok{idea,}
\NormalTok{ideaClusters,}
\NormalTok{agentsReward_}\DecValTok{01}\NormalTok{,}
\NormalTok{agentsReward_}\DecValTok{02}\NormalTok{) }
\KeywordTok{return}\NormalTok{(ideas)}
\NormalTok{\}}
\CommentTok{# Setting up a function to select ideas}
\NormalTok{phase01 <-}\StringTok{ }\ControlFlowTok{function}\NormalTok{(clusters, ideas) \{}
\CommentTok{#Creating a dataframe with Cluster ID, Winner ID and value}
\NormalTok{idClusters <-}\StringTok{ }\KeywordTok{seq}\NormalTok{(}\DecValTok{1}\OperatorTok{:}\NormalTok{clusters)}
\NormalTok{idWinners <-}\StringTok{ }\KeywordTok{rep}\NormalTok{(}\OtherTok{NA}\NormalTok{, clusters)}
\NormalTok{valueWinners <-}\StringTok{ }\KeywordTok{rep}\NormalTok{(}\OtherTok{NA}\NormalTok{, clusters)}
\NormalTok{listWinners <-}\StringTok{  }\KeywordTok{data.frame}\NormalTok{(idClusters, idWinners, valueWinners)}
\NormalTok{lenIdeas <-}\StringTok{ }\KeywordTok{dim}\NormalTok{(ideas)[}\DecValTok{1}\NormalTok{]}
\ControlFlowTok{for}\NormalTok{ (i }\ControlFlowTok{in} \DecValTok{1}\OperatorTok{:}\NormalTok{lenIdeas) \{}
\CommentTok{#Reward the owners of the selected ideas}
\NormalTok{j <-}\StringTok{ }\NormalTok{ideas[i, }\DecValTok{3}\NormalTok{] }\CommentTok{# Check the cluster of the idea}
\ControlFlowTok{if}\NormalTok{ (}\KeywordTok{is.na}\NormalTok{(listWinners[j, }\DecValTok{2}\NormalTok{])) \{}
\CommentTok{# If the cluster of the agent does not have a winner ...}
\CommentTok{# ... (it's value is NA) ...}
\NormalTok{listWinners[j, }\DecValTok{2}\NormalTok{] <-}\StringTok{ }\NormalTok{ideas[i, }\DecValTok{1}\NormalTok{] }\CommentTok{# ... use the agent's ID}
\NormalTok{listWinners[j, }\DecValTok{3}\NormalTok{] <-}\StringTok{ }\NormalTok{ideas[i, }\DecValTok{2}\NormalTok{] }\CommentTok{# ... use the idea's value}
\CommentTok{# rewarding the selected agent by increasing the reward to 1 }
\CommentTok{# (taking into account multiple challenges)}
\NormalTok{ideas[i, }\DecValTok{4}\NormalTok{] <-}
\DecValTok{1} \OperatorTok{+}\StringTok{ }\NormalTok{ideas[i, }\DecValTok{4}\NormalTok{] }
\NormalTok{\}}
\NormalTok{\}}
\KeywordTok{return}\NormalTok{(ideas)}
\NormalTok{\}}
\CommentTok{# Setting up a function to pool ideas}
\NormalTok{phase02 <-}\StringTok{ }\ControlFlowTok{function}\NormalTok{ (selectedIdeas, target) \{}
\KeywordTok{library}\NormalTok{(lpSolve)}
\NormalTok{ideas <-}\StringTok{ }\NormalTok{selectedIdeas}
\CommentTok{#Filtering the list of winners before performing a left join ...}
\CommentTok{# ... with the pooled ideas}
\NormalTok{listWinners <-}
\KeywordTok{filter}\NormalTok{(ideas, agentsReward_}\DecValTok{01} \OperatorTok{>}\StringTok{ }\DecValTok{0}\NormalTok{) }
\CommentTok{# Trying to pool the retained ideas by looking for ...}
\CommentTok{#... the best combination of coefficients}
\NormalTok{objective.in <-}
\NormalTok{listWinners[, }\DecValTok{2}\NormalTok{] }
\CommentTok{# The sum of the pooled ideas ...}
\NormalTok{mat <-}
\KeywordTok{matrix}\NormalTok{(listWinners[, }\DecValTok{2}\NormalTok{], }\DataTypeTok{nrow =} \DecValTok{1}\NormalTok{, }\DataTypeTok{byrow =} \OtherTok{TRUE}\NormalTok{) }
\NormalTok{dir <-}\StringTok{ "<="} \CommentTok{# ... should be below ...}
\NormalTok{rhs <-}\StringTok{ }\NormalTok{target }\CommentTok{# ... the target}
\NormalTok{optimum <-}
\KeywordTok{lp}\NormalTok{(}\DataTypeTok{direction =} \StringTok{"max"}\NormalTok{,}
\NormalTok{objective.in,}
\NormalTok{mat,}
\NormalTok{dir,}
\NormalTok{rhs,}
\DataTypeTok{all.bin =} \OtherTok{TRUE}\NormalTok{) }\CommentTok{# Which is the best combination of pooled ideas?}
\NormalTok{optimum}\OperatorTok{$}\NormalTok{solution }\CommentTok{# The selected ideas}
\NormalTok{listWinners[, }\DecValTok{5}\NormalTok{] <-}\StringTok{ }\NormalTok{optimum}\OperatorTok{$}\NormalTok{solution }\CommentTok{#Assign prizes}
\NormalTok{polledIdeas <-}\StringTok{ }\NormalTok{ideas }\OperatorTok
\KeywordTok{left_join}\NormalTok{(listWinners,}
\DataTypeTok{by =} \KeywordTok{c}\NormalTok{(}\StringTok{"agentsId"}\NormalTok{, }\StringTok{"idea"}\NormalTok{, }\StringTok{"ideaClusters"}\NormalTok{, }\StringTok{"agentsReward_01"}\NormalTok{))}
\CommentTok{# Taking into account previous idea challenges}
\NormalTok{agentsReward_}\DecValTok{02}\NormalTok{ <-}
\NormalTok{polledIdeas}\OperatorTok{$}\NormalTok{agentsReward_}\FloatTok{02.}\NormalTok{x }\OperatorTok{+}\StringTok{ }\NormalTok{polledIdeas}\OperatorTok{$}\NormalTok{agentsReward_}\FloatTok{02.}\NormalTok{y }
\NormalTok{polledIdeas <-}
\KeywordTok{data.frame}\NormalTok{(polledIdeas, agentsReward_}\DecValTok{02}\NormalTok{) }\CommentTok{#add the new column}
\NormalTok{polledIdeas[}\DecValTok{5}\NormalTok{] <-}\StringTok{ }\OtherTok{NULL} \CommentTok{# Removing the redundant columns}
\NormalTok{polledIdeas[}\DecValTok{5}\NormalTok{] <-}\StringTok{ }\OtherTok{NULL} \CommentTok{# Removing the redundant columns}
\KeywordTok{return}\NormalTok{(polledIdeas)}
\NormalTok{\}}
\CommentTok{# Setting up a function to summarize the rewards}
\NormalTok{phase03 <-}\StringTok{ }\ControlFlowTok{function}\NormalTok{(ideas, target) \{}
\KeywordTok{library}\NormalTok{(tidyverse)}
\NormalTok{agentsRewards <-}\StringTok{ }\NormalTok{ideas }\OperatorTok
\KeywordTok{gather}\NormalTok{(}\StringTok{'agentsReward_01'}\NormalTok{,}
\StringTok{'agentsReward_02'}\NormalTok{,}
\DataTypeTok{key =} \StringTok{"Phase"}\NormalTok{,}
\DataTypeTok{value =} \StringTok{"Rewards"}\NormalTok{)}
\NormalTok{winners <-}
\KeywordTok{summarise}\NormalTok{(}\KeywordTok{group_by}\NormalTok{(agentsRewards, agentsId, idea),}
\DataTypeTok{Rewards =} \KeywordTok{sum}\NormalTok{(Rewards, }\DataTypeTok{na.rm =} \OtherTok{TRUE}\NormalTok{))}
\NormalTok{winners <-}\StringTok{ }\KeywordTok{arrange}\NormalTok{(winners, idea) }\CommentTok{#Sorting ideas}
\NormalTok{problemSolved <-}\StringTok{ }\KeywordTok{sum}\NormalTok{(}\KeywordTok{filter}\NormalTok{(winners, Rewards }\OperatorTok{>}\StringTok{ }\DecValTok{1}\NormalTok{)[, }\DecValTok{2}\NormalTok{])}
\KeywordTok{print}\NormalTok{(}\KeywordTok{filter}\NormalTok{(winners, Rewards }\OperatorTok{>}\StringTok{ }\DecValTok{0}\NormalTok{)) }\CommentTok{# Print results}
\KeywordTok{print}\NormalTok{(}\KeywordTok{mean}\NormalTok{(winners}\OperatorTok{$}\NormalTok{Rewards))}
\KeywordTok{print}\NormalTok{(problemSolved)}
\KeywordTok{print}\NormalTok{(target }\OperatorTok{-}\StringTok{ }\NormalTok{problemSolved)}
\KeywordTok{return}\NormalTok{(agentsRewards)}
\NormalTok{\}}
\CommentTok{# Setting up a function to run an idea Challenge}
\NormalTok{ideaChallenge <-}
\ControlFlowTok{function}\NormalTok{(target,}
\NormalTok{totAgents,}
\NormalTok{totIdeas,}
\NormalTok{clusters,}
\NormalTok{sortedIdea) \{}
\NormalTok{generatedIdeas <-}\StringTok{ }\KeywordTok{as.data.frame}\NormalTok{(}\KeywordTok{seq}\NormalTok{(}\DecValTok{1}\OperatorTok{:}\NormalTok{totIdeas))}
\CommentTok{# Generating ideas}
\NormalTok{generatedIdeas <-}
\KeywordTok{phase00}\NormalTok{(target, totAgents, totIdeas, clusters) }
\ControlFlowTok{if}\NormalTok{ (sortedIdea)}
\NormalTok{generatedIdeas <-}
\KeywordTok{arrange}\NormalTok{(generatedIdeas, }\KeywordTok{desc}\NormalTok{(idea)) }\CommentTok{# Sorting ideas by experts}
\NormalTok{selectedIdeas <-}
\KeywordTok{phase01}\NormalTok{(clusters, generatedIdeas) }\CommentTok{# Selecting ideas}
\NormalTok{polledIdeas <-}\StringTok{ }\KeywordTok{phase02}\NormalTok{(selectedIdeas, target) }\CommentTok{# Polling ideas}
\NormalTok{agentsRewards <-}\StringTok{ }\KeywordTok{phase03}\NormalTok{(polledIdeas, target)}
\KeywordTok{return}\NormalTok{(agentsRewards)}
\NormalTok{\}}
\NormalTok{##RUN THE SIMULATION}
\CommentTok{# MODEL 0. Standard idea challenge Benchmark}
\NormalTok{target <-}\StringTok{ }\FloatTok{1e9} \CommentTok{# we set the target at 1'000'000'000}
\NormalTok{totAgents <-}\StringTok{ }\FloatTok{1e3}
\NormalTok{totIdeas <-}\StringTok{ }\FloatTok{1e4}
\NormalTok{clusters <-}\StringTok{ }\DecValTok{1}
\NormalTok{generatedIdeas0 <-}\StringTok{ }\KeywordTok{as.data.frame}\NormalTok{(}\KeywordTok{seq}\NormalTok{(}\DecValTok{1}\OperatorTok{:}\NormalTok{totIdeas))}
\CommentTok{# Generating ideas}
\NormalTok{generatedIdeas0 <-}
\KeywordTok{phase00}\NormalTok{(target, totAgents, totIdeas, clusters) }
\NormalTok{generatedIdeas0 <-}
\KeywordTok{arrange}\NormalTok{(generatedIdeas0, }\KeywordTok{desc}\NormalTok{(idea)) }\CommentTok{#Sorting ideas by experts}
\NormalTok{selectedIdeas0 <-}
\KeywordTok{phase01}\NormalTok{(clusters, generatedIdeas0) }\CommentTok{# Selecting ideas}
\NormalTok{agentsRewards0 <-}\StringTok{ }\NormalTok{selectedIdeas0 }\OperatorTok
\KeywordTok{gather}\NormalTok{(}\StringTok{'agentsReward_01'}\NormalTok{,}
\StringTok{'agentsReward_02'}\NormalTok{,}
\DataTypeTok{key =} \StringTok{"Phase"}\NormalTok{,}
\DataTypeTok{value =} \StringTok{"Rewards"}\NormalTok{)}
\NormalTok{winners <-}\StringTok{ }\KeywordTok{dim}\NormalTok{(}\KeywordTok{filter}\NormalTok{(agentsRewards0, Rewards }\OperatorTok{>}\StringTok{ }\DecValTok{0}\NormalTok{))[}\DecValTok{1}\NormalTok{]}
\NormalTok{quality <-}\StringTok{ }\KeywordTok{filter}\NormalTok{(agentsRewards0, Rewards }\OperatorTok{==}\StringTok{ }\DecValTok{1}\NormalTok{)[, }\DecValTok{2}\NormalTok{]}
\NormalTok{prize <-}\StringTok{ }\KeywordTok{mean}\NormalTok{(agentsRewards0}\OperatorTok{$}\NormalTok{Rewards) }\OperatorTok{*}\StringTok{ }\NormalTok{totAgents}
\NormalTok{challengeID <-}\StringTok{ "Single Winner"}
\NormalTok{performance <-}\StringTok{ }\KeywordTok{data.frame}\NormalTok{(challengeID, winners, quality, prize)}
\NormalTok{performance}\OperatorTok{$}\NormalTok{winners }\CommentTok{# Winners}
\NormalTok{performance}\OperatorTok{$}\NormalTok{quality }\CommentTok{# Sum of the pooled ideas}
\NormalTok{performance}\OperatorTok{$}\NormalTok{prize }\CommentTok{# Cost of prizes}
\CommentTok{# MODEL 1A: Selecting the first idea for each cluster}
\NormalTok{target <-}\StringTok{ }\FloatTok{1e9} \CommentTok{# we set the target at 1'000'000'000}
\NormalTok{totAgents <-}\StringTok{ }\FloatTok{1e3}
\NormalTok{totIdeas <-}\StringTok{ }\FloatTok{1e4}
\NormalTok{clusters <-}\StringTok{ }\DecValTok{10}
\NormalTok{agentsRewards1_}\DecValTok{1}\NormalTok{ <-}
\KeywordTok{ideaChallenge}\NormalTok{(target, totAgents, totIdeas, clusters, }\OtherTok{FALSE}\NormalTok{)}
\CommentTok{# Model 1B: Selecting the idea for each cluster with ...}
\CommentTok{#...the highest score}

\CommentTok{#using sorted ideas}
\NormalTok{agentsRewards2 <-}
\KeywordTok{ideaChallenge}\NormalTok{(target, totAgents, totIdeas, clusters, }\OtherTok{TRUE}\NormalTok{) }
\CommentTok{# MODEL 2: Cheatstorming}
\NormalTok{target <-}\StringTok{ }\NormalTok{remainder1_}\DecValTok{1} \CommentTok{# New target}
\NormalTok{generatedIdeas <-}\StringTok{ }\NormalTok{agentsRewards1_}\DecValTok{1} \OperatorTok
\KeywordTok{spread}\NormalTok{(}\DataTypeTok{key =}\NormalTok{ Phase, }\DataTypeTok{value =}\NormalTok{ Rewards)  }\CommentTok{# Spread the table}
\NormalTok{generatedIdeas <-}
\KeywordTok{filter}\NormalTok{(generatedIdeas, idea }\OperatorTok{<}\StringTok{ }\NormalTok{target) }\CommentTok{#Select ideas}
\CommentTok{# New clusters}
\NormalTok{generatedIdeas}\OperatorTok{$}\NormalTok{ideaClusters <-}
\DecValTok{1} \OperatorTok{+}\StringTok{ }\KeywordTok{as.integer}\NormalTok{(generatedIdeas}\OperatorTok{$}\NormalTok{idea }\OperatorTok{/}\StringTok{ }\NormalTok{(target }\OperatorTok{/}\StringTok{ }\NormalTok{clusters))}
\CommentTok{# Reward all contributions in phase 01}
\NormalTok{generatedIdeas}\OperatorTok{$}\NormalTok{agentsReward_}\DecValTok{01}\NormalTok{ <-}
\DecValTok{1} \OperatorTok{/}\StringTok{ }\KeywordTok{dim}\NormalTok{(generatedIdeas)[}\DecValTok{1}\NormalTok{] }\OperatorTok{+}\StringTok{ }\NormalTok{generatedIdeas}\OperatorTok{$}\NormalTok{agentsReward_}\DecValTok{01} 
\ControlFlowTok{if}\NormalTok{ (}\KeywordTok{is.na}\NormalTok{(generatedIdeas}\OperatorTok{$}\NormalTok{agentsReward_}\DecValTok{02}\NormalTok{)) \{}
\NormalTok{generatedIdeas}\OperatorTok{$}\NormalTok{agentsReward_}\DecValTok{02}\NormalTok{ <-}\StringTok{ }\DecValTok{0} \CommentTok{# Remove NA from column 2}
\NormalTok{\}}
\NormalTok{selectedIdeas <-}\StringTok{ }\NormalTok{generatedIdeas}
\NormalTok{pooledIdeas <-}\StringTok{ }\KeywordTok{phase02}\NormalTok{(selectedIdeas, target) }\CommentTok{# Pooling ideas}
\CommentTok{# Check number of winners of phase}
\NormalTok{agentsRewards <-}\StringTok{ }\KeywordTok{phase03}\NormalTok{(pooledIdeas, target)}
\KeywordTok{filter}\NormalTok{(agentsRewards, Rewards }\OperatorTok{>=}\StringTok{ }\DecValTok{1}\NormalTok{) }
\end{Highlighting}
\end{Shaded}

\newpage

\section*{8 References}\label{references}
\addcontentsline{toc}{section}{8 References}

\hypertarget{refs}{}
\hypertarget{ref-afuah_crowdsourcing_2012}{}
Afuah, Allan, and Christopher L. Tucci. 2012. ``Crowdsourcing as a
Solution to Distant Search.'' \emph{Academy of Management Review} 37
(3): 355--75.

\hypertarget{ref-boudreau_using_2013}{}
Boudreau, Kevin J., and Karim R. Lakhani. 2013. ``Using the Crowd as an
Innovation Partner.'' \emph{Harvard Business Review} 91 (4): 60--69.

\hypertarget{ref-bullinger_community-based_2010-1}{}
Bullinger, Angelika C., Anne-Katrin Neyer, Matthias Rass, and Kathrin M.
Moeslein. 2010. ``Community-Based Innovation Contests: Where Competition
Meets Cooperation.'' \emph{Creativity and Innovation Management} 19 (3):
290--303.

\hypertarget{ref-chesbrough_explicating_2014}{}
Chesbrough, Henry, and M. Borgers. 2014. ``Explicating Open Innovation:
Clarifying an Emerging Paradigm for Understanding Innovation.'' In
\emph{New Frontiers in Open Innovation}. Oup Oxford.

\hypertarget{ref-chesbrough_open_2006}{}
Chesbrough, Henry, Wim Vanhaverbeke, and J West. 2006. \emph{Open
Innovation: Researching a New Paradigm}. Oxford: New York: Oxford Univ.
Press. \url{https://doi.org/10.1111/j.1467-8691.2008.00502.x}.

\hypertarget{ref-farmer_economy_2009}{}
Farmer, J. Doyne, and Duncan Foley. 2009. ``The Economy Needs
Agent-Based Modelling.'' \emph{Nature} 460 (7256): 685.

\hypertarget{ref-faste_brainstorm_2013}{}
Faste, Haakon, Nir Rachmel, Russell Essary, and Evan Sheehan. 2013.
``Brainstorm, Chainstorm, Cheatstorm, Tweetstorm.'' In \emph{Proceedings
of the SIGCHI Conference on Human Factors in Computing Systems - CHI
'13}. \url{https://doi.org/10.1145/2470654.2466177}.

\hypertarget{ref-felin_firms_2017}{}
Felin, Teppo, Karim R. Lakhani, and Michael L. Tushman. 2017. ``Firms,
Crowds, and Innovation.'' \emph{Strategic Organization} 15 (2): 119--40.

\hypertarget{ref-gregor_positioning_2013}{}
Gregor, S., and A.R. Hevner. 2013. ``Positioning and Presenting Design
Science Research for Maximum Impact.'' \emph{MIS Quarterly} 37 (2):
337--55.

\hypertarget{ref-halaburda_blockchain_2018}{}
Halaburda, H. 2018. ``Blockchain Revolution Without the Blockchain?''
\emph{Communication of the ACM,} 61 (7): 27--29.

\hypertarget{ref-hevner_design_2004}{}
Hevner, Alan, Salvatore March, Jinsoo Park, and Sudha Ram. 2004.
``Design Science in Information Systems Research.'' \emph{Management
Information Systems Quarterly} 28 (1).
\url{http://aisel.aisnet.org/misq/vol28/iss1/6}.

\hypertarget{ref-lakhani_open_2012}{}
Lakhani, Karim, Hila Lifshitz-Assaf, and Michael Tushman. 2012. ``Open
Innovation and Organizational Boundaries: The Impact of Task
Decomposition and Knowledge Distribution on the Locus of Innovation.''

\hypertarget{ref-malone_handbook_2015}{}
Malone, Thomas W., and Michael S. Bernstein. 2015. \emph{Handbook of
Collective Intelligence}. MIT Press.

\hypertarget{ref-march_design_1995}{}
March, Salvatore T., and Gerald F. Smith. 1995. ``Design and Natural
Science Research on Information Technology.'' \emph{Decision Support
Systems} 15 (4): 251--66.

\hypertarget{ref-nakamoto_bitcoin:_2008}{}
Nakamoto, S. 2008. ``Bitcoin: A Peer-to-Peer Electronic Cash System.''
\url{https://doi.org/10.1007/s10838-008-9062-0}.

\hypertarget{ref-pries-heje_strategies_2008}{}
Pries-Heje, Jan, Richard Baskerville, and John R. Venable. 2008.
``Strategies for Design Science Research Evaluation.'' In \emph{ECIS},
255--66.

\hypertarget{ref-riedl_teams_2017}{}
Riedl, Christoph, and Anita Williams Woolley. 2017. ``Teams Vs. Crowds:
A Field Test of the Relative Contribution of Incentives, Member Ability,
and Emergent Collaboration to Crowd-Based Problem Solving Performance.''
\emph{Academy of Management Discoveries} 3 (4): 382--403.

\hypertarget{ref-shrier_blockchain_2016}{}
Shrier, D., W. Wu, and A. Pentland. 2016. ``Blockchain \& Infrastructure
(Identity,data Security).'' \emph{MIT Connection Science \&
Engineering,} May (3): 1--19.

\hypertarget{ref-sun_cross-evaluation:_2006}{}
Sun, Ying, and Paul B. Kantor. 2006. ``Cross-Evaluation: A New Model for
Information System Evaluation.'' \emph{Journal of the American Society
for Information Science and Technology} 57 (5): 614--28.

\hypertarget{ref-swan_blockchain_2015}{}
Swan, M. 2015. ``Blockchain Thinking\,: The Brain as a Decentralized
Autonomous Corporation.'' \emph{IEEE Technology and Society Magazine}.
\url{https://doi.org/10.1109/MTS.2015.2494358}.

\hypertarget{ref-tapscott_this_2017}{}
Tapscott, D. 2017. ``This Is How Block Chain Will Change Your Life.''
\url{https://www.weforum.org/agenda/2016/06/this-is-how-blockchain-will-change-your-life/}.

\hypertarget{ref-tapscott_blockchain_2016}{}
Tapscott, D, and A Tapscott. 2016. \emph{Blockchain Revolution: How the
Technology Behind Bitcoin Is Changing Money, Business, and the World}.
Portfolio.

\hypertarget{ref-r._core_team_r:_2013}{}
Team, R. Core. 2013. ``R: A Language and Environment for Statistical
Computing.''

\hypertarget{ref-tilson_research_2010}{}
Tilson, David, Kalle Lyytinen, and Carsten Sørensen. 2010. ``Research
Commentary---Digital Infrastructures: The Missing IS Research Agenda.''
\emph{Information Systems Research} 21 (4): 748--59.

\hypertarget{ref-tucci_creating_2018}{}
Tucci, Christopher L., Allan Afuah, and Gianluigi Viscusi. 2018.
\emph{Creating and Capturing Value Through Crowdsourcing}. Oxford
University Press.

\hypertarget{ref-vanhaverbeke_classification_2014}{}
Vanhaverbeke, Wim, and Henry Chesbrough. 2014. ``A Classification of
Open Innovation and Open Business Models.'' In \emph{New Frontiers in
Open Innovation}, edited by Henry Chesbrough, Wim Vanhaverbeke, and J
West, 50--68.

\hypertarget{ref-venable_comprehensive_2012}{}
Venable, John R., Jan Pries-Heje, and Richard Baskerville. 2012. ``A
Comprehensive Framework for Evaluation in Design Science Research.'' In
\emph{International Conference on Design Science Research in Information
Systems}, 423--38. Springer.

\hypertarget{ref-viscusi_open_2015}{}
Viscusi, Gianluigi, Diane Poulin, and Christopher Tucci. 2015. ``Open
Innovation Research and E-Government: Clarifying the Connections Between
Two Fields.'' In \emph{Proceedings of the XII Conference of the Italian
Chapter of AIS (itAIS2015)}. Luiss University Press.

\hypertarget{ref-woolley_collective_2015}{}
Woolley, Anita Williams, Ishani Aggarwal, and Thomas W. Malone. 2015.
``Collective Intelligence and Group Performance.'' \emph{Current
Directions in Psychological Science} 24 (6): 420--24.

\hypertarget{ref-yoo_tables_2013}{}
Yoo, Youngjin. 2013. ``The Tables Have Turned: How Can the Information
Systems Field Contribute to Technology and Innovation Management
Research?'' \emph{Journal of the Association for Information Systems} 14
(5): 227.

\end{document}


